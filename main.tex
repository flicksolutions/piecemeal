\documentclass{article}
\usepackage[utf8]{inputenc}
%\usepackage{german}
\usepackage{color} %Zum stylen von \def
\usepackage{varwidth} %Zum stylen von \def
\usepackage{etoolbox} %Zum stylen von \def
\usepackage{amssymb} %Für die Mengensymbole
\usepackage{svg}
\usepackage{amsmath} %Zur Darstellung von Funktionen mit Fallunterscheidung
\usepackage{scrlayer-scrpage} %Kopf- und Fusszeile
\usepackage{titling} %Titel nach oben verschieben
\usepackage{amsfonts} %\mathcal
\usepackage{enumerate} %Für kleine römische Zahlen in Aufzählungen


%\usepackage{prettyref}
%\usepackage{apacite}
\usepackage[
backend=biber,
style=apa,
]{biblatex}
\addbibresource{references.bib}

%\usepackage{array,multirow}



%% Formatierung %%
\usepackage[dvipdfm]{geometry}
\setlength{\droptitle}{-10em}
\geometry{bottom=3.5cm, textwidth=15cm, headsep=15mm, top=5cm,head=15pt}
%%%%%%%%%%%%%%%%%%

%\usepackage{vaucanson-g}
%%  Kopf- und Fusszeile anpassen %%
\pagestyle{scrheadings} 

\ihead{\small{Bachelorarbeit}}
\ohead{\small{Sebastian Flick, Universität Bern}}
\cfoot{\thepage}
%%%%%%%%%%%%%%%%%%%%%%%%%%%%%%%%%%%

\newcounter{std}
\setcounter{std}{1}

%% New Commands %%
\newcommand{\comment}[1]{{\color{red} #1}}

\definecolor{bg}{gray}{1}
\newcommand{\spec}[3]{
    \begin{flushleft}
    \mpar{\vspace{.5cm}#2}\colorbox{bg}{\hspace{.4cm}\parbox{\textwidth}{
        \begin{varwidth}{\dimexpr\linewidth-2\fboxsep}
            \vspace*{.2cm}
            \noindent{\textbf{#1 \ifstrempty{#2}{}{\textit{#2}: }}#3}
            \stepcounter{std}
        \end{varwidth}
    }}
    \end{flushleft}
}

\newcommand{\formula}[1]{\\$#1$}

\renewcommand{\L}{\Box}
\newcommand{\M}{\Diamond}
\newcommand{\qed}{\hspace{0.5cm}$\blacksquare$}

\newcommand\mpar[1]{\marginpar {\flushleft\sffamily\small #1}}
\setlength{\marginparwidth}{2cm}

%%%%%%%%%%%%%%%%%%%%

\newcommand{\R}{\mathbb{R}}
\newcommand{\Z}{\mathbb{Z}}
\newcommand{\N}{\mathbb{N}}
\newcommand{\C}{\mathbb{C}}

\newcommand{\bmt}{\begin{pmatrix}}
\newcommand{\emt}{\end{pmatrix}}

\title{Ein realistischeres Modell des Reflective Equilibrium}
\author{Bachelorarbeit von Sebastian Flick, 16-121-014\\Betreut von Prof. Dr. Dr. Claus Beisbart\\Universität Bern}


\date{\today}

\begin{document}
\maketitle
\section{Einführung}
Das Ziel der Arbeit ist es, das Modell von \autocite{beisbart_making_2015} um einen kleinen aber wichtigen Aspekt zu verändern: Die Anpassung von Überzeugungen und Theorien umsetzbarer und nachvollziehbarer zu machen. Bisher funktioniert der Anpassungsschritt der Überzeugungen an eine Theorie bzw. der Theorie an eine Menge von Überzeugungen in einem komplexen, schwer nachvollziehbaren Schritt: Jede kombinatorisch mögliche Menge an Überzeugungen wird mit der aktuellen Theorie verglichen und dann wird die Beste ausgewählt. Dies ist einerseits nicht, was \citeauthor{goodman_fact_1983} im Sinn hatte, als er Reflective Equilibrium beschrieb, andererseits ab einer gewissen grösse des zu behandelnen Themas schlicht nicht umsetzbar. Deshalb werden in dieser Arbeit piecemeal-Ansätze erläutert, welche die Vorgehensweise nachvollziehbarer und automatische Systeme leistungsfähiger machen sollen.

Ich werde zum Beginn der Arbeit die Methode Reflective Equilibrium vorstellen und eine kleine historische Einführung in das Thema geben. Danach werde ich zwei Ansätze vorstellen, welche die Anpassung piecemeal-artig angehen. Ich werde sie innerhalb des Modells von \citeauthor{beisbart_making_2015} definieren und beweisen, dass die Ansätze effizienter sind als der konventionelle.Anhand von Simulationen werde ich aufzeigen, dass der Ansatz umsetzbar ist.

\section{Die Ursprünge von Reflective Equilibrium}

\begin{quote}
    A rule is amended if it yields an inference we are unwilling to accept; an inference is rejected if it violates a rule we are unwililng to amend. \autocite[S.~64]{goodman_fact_1983}
\end{quote}

\section{Umsetzung}
\subsection{Das bisherige System}

\begin{figure}[htbp]
  \centering
  \includesvg{figure1}
  \caption{eine kleine Menge von Sätzen\label{fig:smallset}}
\end{figure}

Für eine sehr kleine Menge von Sätzen gelingt eine übersichtliche Darstellung noch. Bei Abbildung \ref{fig:smallset} ist eine kleine Menge von Sätzen abgebildet. Wenn nun \textit{s3} das einzige Commitment ist und man dazu die passende Theorie suchen möchte, so würden folgende Kombinationen überprüft: $\{s1, s2, s3, \{s1,s3\}, \{s2,s3\}\}$ Lediglich die Menge $\{s1,s2\}$ kann ausgeschlossen werden, da sie inkonsistent ist. Die restlichen 5 Elemente müssten überprüft werden. Wenn nun ein einziges weiteres Element \textit{s4} hinzukommt, muss folgende Menge überprüft werden:\linebreak
$\{s1, s2, s3, s4, \{s1,s3\}, \{s1,s4\}, \{s2,s3\}, \{s2,s4\}, \{s1,s3,s4\}, \{s2,s3,s4\}\}$. Es handelt sich um 10 Elemente. Das heisst, der Arbeitsaufwand wächst exponentiell zur Anzahl an Sätzen im System. Dies macht das System unrealistisch in der Benutzung.



\subsection{Der neue Ansatz}

\subsubsection{Einzelne Elemente hinzufügen oder entfernen}
Wie \cite[S.25]{beisbart_making_2015} schon feststellten, scheint die einfachste Lösung zu sein, einzelne Commitments oder Prinzipien zu ändern und zumindest \citeauthor{goodman_fact_1983} spricht sich klar für einen piecemeal-Ansatz aus. Der erste Ansatz dreht sich um die Idee, von den bestehenden Überzeugungen bzw. der bestehenden Theorie auszugehen anstatt agnostisch davon auszugehen, dass sich alle möglichen Theorien gleich gut eignen. Man ändert die bestehende Theorie bzw. die bestehenden Commitments um ein einzelnes Element ab und sieht, ob diese abgeänderte Theorie bzw. die abgeänderten Überzeugungen besser zu ihrem Gegenstück passen.

\paragraph{Anpassung der Theorie}
Ausgehend von einer Menge von Überzeugungen $C_i$ wird eine Theorie $T_i$ wie folgt angepasst: 
\begin{enumerate}
    \item \label{1} Alle Sätze des Themas $S$ werden einzeln zu der Menge von Prinzipien in $T_i$ hinzugefügt, falls sie noch nicht enthalten sind und die resultierende Theorie konsistent ist. Die Menge der daraus entstandenen Theorien heisst $T_i^*$.
    \item \label{2}Die Mengen, die entstehen, wenn man einzelne Prinzipien aus $T_i$ entfernt, werden zu $T_i^*$ hinzugefügt.
    \item \label{3}Die Elemente der Menge der Theorien $T_i^*$ wird mittels Achievement-Funktion $Z$ überprüft.
    \item \label{4}Die Theorie, welche den höchsten Wert bei der $Z$ erreicht, wird die Theorie $T_i$ ersetzen und zu $T_{i+1}$ erklärt.
\end{enumerate}

\paragraph{Anpassung der Überzeugungen}
Ausgehend von einer Theorie $T_i$ wird die Menge der Überzeugungen $C_i$ wie folgt angepasst:
\begin{enumerate}
    \item \label{c1} Alle Sätze des Themas $S$ werden einzeln zu der Menge von Überzeugungen in $C_i$ hinzugefügt, falls sie noch nicht enthalten sind. Falls sie aber enthalten sind, wird die nur Negation des Satzes hinzugefügt. Die Menge der daraus entstandenen Mengen von Überzeugungen heisst $C_i^*$.
    \item \label{c2} Die Mengen, die entstehen, wenn man einzelne Überzeugungen aus $C_i$ entfernt, werden zu $C_i^*$ hinzugefügt.
    \item \label{c3} Die Elemente der Menge der Theorien $T_i^*$ wird mittels Achievement-Funktion $Z$ überprüft.
    \item \label{c4}Die Menge von Überzeugungen, welche den höchsten Wert bei der $Z$ erreicht, wird die Menge $C_i$ ersetzen und zu $C_{i+1}$ erklärt.
\end{enumerate}

\begin{figure}
  \centering
  \includesvg{classical}
  \caption{Die klassische Struktur im initialen Zustand A\label{fig:classset-initial1}}
\end{figure}

 Als Beispiel für den Ablauf kann die klassische Struktur \ref{fig:classset-initial1} im initialen Zustand mit $C_0 = \{s3, s4, s5\}$ zur Hand genommen werden. Die erste Anpassung der Theorie überprüft nun einfach alle einzelnen Sätze $\{s1,...,s7\}$. Derjenige welche alleine den höchsten Wert von $Z$ erreicht, wird zum einzigen Prinzip. Dies ist $s1$, weil er alle Überzeugungen rechtfertigt und damit wird $\{s1\}$ zu $T_0$.
 
 Die Überzeugungen müssen nun angepasst werden. Dazu überprüfen wir die Menge von Mengen welche $C_0$ und einen weiteren Satz enthalten. Im Schritt \ref{c1} werden ausserdem alle Mengen erstellt, die entstehen, wenn man einzelne Sätze in $C_0$ negiert, zum Beispiel $\{\lnot s3, s4, s5\}$. Ausserdem überprüfen wir entsprechend Schritt \ref{c2} die Mengen, welche entstehen, wenn man einen einzelnen Satz aus $C_0$ entfernt. Die Überzeugungen werden dahingehend verändert, dass entweder $\neg s6$ oder $neg s2$ hinzugefügt werden, weil dadurch der Parameter \textit{Account} verbessert wird. Dieser wird bei beiden Änderungen im selben Mass verändert, das heisst, es muss eine Entscheidung getroffen werden. Zur Einfachheit des Beispiels gehe ich davon aus, dass zufällig eine die Möglichkeit gewählt wird, $neg s2$ zu den Überzeugungen hinzuzufügen. Ich komme allerdings im Abschnitt \ref{branching} auf eine weitere Möglichkeit zu sprechen. $C_1 = \{s3, s4, s5, \neg s2\}$
 
 Bei der zweiten Anpassung der Theorie müssen nun Theorien geprüft werden, welche $s1$ und einen weiteren Satz enthalten. Hierbei kann $\{s1,s2\}$ ausgeschlossen werden, weil die Theorie inkonsistent ist. Ausserdem muss die leere Theorie überprüft werden - denn sie resultiert aus Schritt \ref{2}. Das Resultat ist, dass keine der überprüften Theorien besser ist als $T_1$. Somit wird $T_1$ auch zu $T_2$ übernommen.
 
 Es folgt eine weitere Anpassung der Überzeugungen. Da sich die Theorie nicht verändert hat, liegt auf der Hand, dass das Hinzufügen von $\neg s6$ zu $C_1$ die beste Alternative ist. Somit sieht $C_2$ nun wie folgt aus: $C_2 = \{s3, s4, s5, \neg s2, \neg s6\}$
 
 Weil $T_2$ mit $T_1$ identisch ist, müssen auch dieselben Theorien überprüft werden. Das Ergebnis wird sich nicht verändern und $T_3$ wird mit $T_2$ gleichgesetzt.
 
 Es folgt eine letzte Überprüfung der Überzeugungen. Nach dem beschriebenen Rezept wird diese wieder durchgeführt, mit dem Ergebnis, dass es keine Verbesserung gibt und $C_3$ mit $C_4$ gleichgesetzt wird. Damit endet der Prozess mit dem Ergebnis eines Reflective Equilibriums.
 
 Während des Prozesses mussten in der ersten Änderung sieben Theorien und in der zweiten und dritten Änderung sechs Theorien überprüft und verglichen werden.  Im Vergleich zum traditionellen Ansatz, bei dem in jedem Schritt $2^7$ Theorien (ohne Berücksichtigung der inkonsistenten Theorien) überprüft und verglichen werden. In der Anpassung der Überzeugungen müssen bei jedem Schritt immer $2n + (t-n)$ Überzeugungen überprüft werden. Wobei $n$ der Anzahl Überzeugungen in $C_i$ entspricht und $t$ der Anzahl von Sätzen im Topic. Für das Beispiel heisst das für Schritt Eins $2*3 + (7-3) = 10$, für Schritt Zwei $11$ und für Schritt Drei und Vier $12$. Beim traditionellen Ansatz müssen bei jedem Schritt $3^7$ Überzeugungen überprüft werden - weil ein Satz und dessen Negation überprüft werden müssen, erhöht sich die Basis im Vergleich zur Theorieanpassung. Wichtig für die Frage, ob der traditionelle Ansatz wirklich weniger effizient ist, ist neben der Anzahl zu überprüfenden Mengen pro Schritt auch die Anzahl der Schritte. Diese wäre für unser Beispiel für das traditionelle Modell 4 Schritte (2 für die Überzeugungen und 2 für die Theorien), für den neuen Ansatz sind es 7 Schritte. Die Anzahl Schritte ist 1.75 Mal höher, aber die Anzahl von Rechenschritten pro Schritt ist $51.\Bar{4}$ Mal höher. Wenn sich bei weiteren Kompositionen ähnliche Werte ergeben, ist die neue Methode der traditionellen vorzuziehen. Ausserdem gilt es zu überprüfen, ob die Methode globale Optima erreicht.
 % Run the numbers again... erhöht sich die Basis bei Commitmentanpassung wirklich um 1?

\subsubsection{Analyse der Inferenzbeziehungen}
Eine weitere Idee, einen piecemeal-Ansatz umzusetzen, ist die Analyse der gegebenen Inferenzbeziehungen. Aufgrund der Inferenzbeziehungen zwischen Sätzen, kann klug entschieden werden, welche Sätze sich eher als Theorien lohnen. Auch für die Anpassung der Überzeugungen kann es hilfreich sein, zu analysieren, welche Inferenzbeziehungen die zuletzt hinzugefügten Prinzipien haben.

\paragraph{Anpassung der Theorie}
Ausgehend von einer Menge von Überzeugungen $C_i$ wird eine Theorie $T_i$ wie folgt angepasst. Hierbei entspricht $z$ der Anzahl Durchläufe und $c_a$ der aktuell ausgewählten Überzeugung. 
\begin{enumerate}
    \item \label{i1} Die erste Überzeugung aus $C_i$ wird ausgewählt: $c_a = c_0$ (Index 0)
    \item \label{i2} Falls die Menge $S_z$ leer ist: Für die ausgewählte Überzeugung $c_a$ werden alle Prinzipien gesucht, welche die ausgewählte Überzeugung zum Inhalt haben. Jedes gefundene Prinzip eröffnet eine neue Menge und diese Mengen werden $S$ hinzugefügt. Falls $S_z$ nicht leer ist: Jedes für $c_a$ gefundene Prinzip wird zu den Mengen in $S_z$ hinzugefügt, falls die entstehende Menge konsistent ist.
    \item \label{i3} Falls $c_x$ noch nicht das letzte Element aus $C_i$ ist, wird die nächste Überzeugung aus $C_i$ wird ausgewählt und zu Schritt \ref{i2} gesprungen.
    \item \label{i4} Falls $z < \lvert C_i \rvert$: Die Reihenfolge der Überzeugungen in $C_i$ wird geändert. Dabei wird die erste Überzeugung zur letzten. $z$ wird um eins inkrementiert: $z = z + 1$ und zu Schritt \ref{i1} gesprungen.
    \item \label{i5} In den Mengen $S_\alpha$ werden die Mengen, welche die gleichen Elemente enthalten zu einer reduziert. Alle übriggebliebenen Mengen werden mittels Achievement-Funktion $Z$ überprüft.
    \item \label{i6}Die Theorie, welche den höchsten Wert bei $Z$ erreicht, wird die Theorie $T_i$ ersetzen und zu $T_{i+1}$ erklärt, ausser ihr Wert ist kleiner als der Wert von $T_i$. In diesem Fall wird $T_i$ zu $T_{i+1}$ erklärt.
\end{enumerate}

\paragraph{Anpassung der Überzeugungen}
Ausgehend von einer Theorie $T_i$ werden für $C_i+1$ die Prinzipien und der Inhalt von $T_i$ übernommen.
Durch diesen Schritt wird immer voller \textit{Account} garantiert, dafür wird $C_0$ aber vernachlässigt und somit \textit{Faithfulness} gefährdet. Es muss durch Experimente gezeigt werden, ob dieser Ansatz zu gleich guten Ergebnissen führt wie der traditionelle Ansatz.

\newpage
\printbibliography

\end{document}


%%%%%%%%%%%%%%%%%%%%%%%%


\begin{enumerate}
    \item How can logic be defined in a general way, so that different rival theories can be understood as \textit{logical} theories?
\end{enumerate}

\begin{quote}
    An argument is logically valid just when conditional [sic.], with the conjunction of the premises as antecendent and the conclusion as consequent, is logically true. We can readily slip from one notion to the other[...] \cite[p.~13]{beallrestall}
\end{quote}

\subsection{Debating on logic}\label{sec:debating}
\footnote{An extensive discussion of trivialism and why it is an irrational view can be found in chapter 3 of \citeA{priest2}.}

 \fullciteauthor{beallrestall}

(see section \ref{principle:GGTP}, page \pageref{principle:GGTP})

\begin{center}
\begin{tabular}{c p{9cm}}
P1 & If I am king of France, I'm not king of France.\\\hline
C & If I am king of France, God exists. 
\end{tabular}
\end{center}

\paragraph{Systematicity} To define the consequence relation by specifying it for each argument separately would be a very unsystematic procedure. \textbf{Systematicity}, like the one brought forward by \citeauthor{resnik0} or \citeauthor{peregrinsvoboda}. They have, as I will argue, a problematic understanding of the purpose of an RE-process.