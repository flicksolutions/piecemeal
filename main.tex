\documentclass{article}
\usepackage[utf8]{inputenc}
%\usepackage{german}
\usepackage{color} %Zum stylen von \def
\usepackage{varwidth} %Zum stylen von \def
\usepackage{etoolbox} %Zum stylen von \def
\usepackage{amssymb} %Für die Mengensymbole
\usepackage{amsmath} %Zur Darstellung von Funktionen mit Fallunterscheidung
\usepackage{scrlayer-scrpage} %Kopf- und Fusszeile
\usepackage{titling} %Titel nach oben verschieben
\usepackage{amsfonts} %\mathcal
\usepackage{enumerate} %Für kleine römische Zahlen in Aufzählungen
\usepackage{prettyref}
\usepackage{tikz}
\usepackage{apacite}
\usepackage{array,multirow}
%\usepackage{ulem} % zum druchstreichen von Text mit \sout{} bzw. \xout{}

% to display strict implication
\DeclareSymbolFont{symbolsC}{U}{txsyc}{m}{n}
\DeclareMathSymbol{\strictif}{\mathrel}{symbolsC}{74}

\DeclareSymbolFont{fgerm}{U}{fgerm}{m}{n}
\DeclareMathSymbol{\fgecap}{\mathord}{fgerm}{53}

\usepackage[arrow, matrix, curve]{xy} % Paket für kommutative Diagramme, hier verwendet zur Darstellung der Semantik

\usepackage[onehalfspacing]{setspace} % Zeilenabstand 1.5
\usepackage[default]{gfsbodoni}
\usepackage[T1]{fontenc}
%\usepackage{vaucanson-g}

%% Formatierung %%
\usepackage[dvipdfm]{geometry}
\setlength{\droptitle}{-10em}
\geometry{bottom=3.5cm, textwidth=15cm, textheight=20cm, headsep=15mm, top=5cm,}
%%%%%%%%%%%%%%%%%%

%\usepackage{vaucanson-g}
%%  Kopf- und Fusszeile anpassen %%
\pagestyle{scrheadings} 

\ihead{\small{Bachelorarbeit}}
\ohead{\small{Sebastian Flick, Universität Bern}}
\cfoot{\thepage}
%%%%%%%%%%%%%%%%%%%%%%%%%%%%%%%%%%%

\newcounter{std}
\setcounter{std}{1}

%% New Commands %%
\newcommand{\comment}[1]{{\color{red} #1}}

\definecolor{bg}{gray}{1}
\newcommand{\spec}[3]{
    \begin{flushleft}
    \mpar{\vspace{.5cm}#2}\colorbox{bg}{\hspace{.4cm}\parbox{\textwidth}{
        \begin{varwidth}{\dimexpr\linewidth-2\fboxsep}
            \vspace*{.2cm}
            \noindent{\textbf{#1 \ifstrempty{#2}{}{\textit{#2}: }}#3}
            \stepcounter{std}
        \end{varwidth}
    }}
    \end{flushleft}
}

\newcommand{\formula}[1]{\\$#1$}

\renewcommand{\L}{\Box}
\newcommand{\M}{\Diamond}
\newcommand{\qed}{\hspace{0.5cm}$\blacksquare$}

\newcommand\mpar[1]{\marginpar {\flushleft\sffamily\small #1}}
\setlength{\marginparwidth}{2cm}

%%%%%%%%%%%%%%%%%%%%

\newcommand{\R}{\mathbb{R}}
\newcommand{\Z}{\mathbb{Z}}
\newcommand{\N}{\mathbb{N}}
\newcommand{\C}{\mathbb{C}}

\newcommand{\bmt}{\begin{pmatrix}}
\newcommand{\emt}{\end{pmatrix}}

\title{Ein realistischeres Modell des Reflective Equilibrium}
\author{Bachelorarbeit von Sebastian Flick, 16-121-014\\Betreut von Prof. Dr. Dr. Claus Beisbart\\Universität Bern}
\date{\today}

\begin{document}
\maketitle
\section{Introduction}
There is a lively philosophical debate on different logical theories. Some philosophers argue that standard logic is wrong in embracing the principle of explosion and we better use a paraconsistent logic. Dialetheists agree, because they think that there are true contradictions. Some of them also claim that, for this reason, our logical theory should be strong enough to prove these contradictions to be true. Others, like intuitionists, may question not only that there are true contradictions, but also that every proposition is either true or false. To defend their views on logic, participants of this debate often support their position with subtle arguments and argue against other positions.

I think that it would not be intellectually honest to simply reject all these theories but one as irrational and I will argue that we have to accept that there is more than one rational logical theory. However, this raises several questions.

\begin{enumerate}
    \item How can logic be defined in a general way, so that different rival theories can be understood as \textit{logical} theories?
    \item How can a logical theory be shown to be rational? Theories of rationality often presuppose a logical theory or refer to logical notions, but this seems to be problematic when we ask whether a logical theory itself is rational.
    \item Are there notions and concepts that are shared among the opponents in the debate on logic and, if there are such notions and concepts, how can they help to answer the previous question of rationality?
\end{enumerate}

This essay aims at answering these questions. To do this, I will make use of the theory of \textit{reflective equilibrium} (\textit{RE}). The idea to justify a logical theory this way goes back to \citeA{Goodman} and was later brought forward by \citeA{resnik1}. This approach was criticised by \citeA{shapiro}. A further goal of this essay is therefore to develop an account of RE that can solve the problems, pointed out by Shapiro.

I will start in section 2 with a definition of the term `logic'. This will include the characterisation of what I think is the `core business' of logic and a discussion of notions like `logical consequence'. In section 3, I will then move on to the central problems of this essay, namely to the problem of what I call `moderate logical pluralism' and the question of how a logical theory can be justified. Furthermore, the theory of RE will be introduced. I will to explain how RE can be used to show that a logical theory is rational. In section 4 I will be concerned with working out a more detailed account of RE. The goal of this is to show that the central terms can be defined in a way that is independent of specific logical theories. I will also discuss principles of rationality that must be considered to use RE to show that a logical theory is rational. In section 5, I will discuss Shapiro's critique of Resnik's idea to apply RE-theory in the philosophy of logic. I will thereby criticise some solutions brought forward in reaction to Shapiros critique and explain how the version of RE I developed solves the problem.

\section{What is logic?}\label{sec:logic}
Logic is here understood as a discipline concerned with the study of arguments and the question which arguments are valid and which are not. A logical theory uses a formal system, including a formal language to represent the abstract form of arguments, and a definition of a consequence relation to specify valid arguments. Since logical theories are used to judge whether an argument is valid or not, they are normative. I cannot provide a detailed discussion of this view. Nevertheless, I will explain briefly why I think this is an accurate view.

The first point that might be controversial is that logic is primarily concerned with the validity of arguments. \citeA{beallrestall} for example, point out that Frege and Russell regarded logic as the study of logical truth \cite[p.~13]{beallrestall}. Beall and Restall argue that logic can as well be understood as the study of logical truth, since logical truths can be translated in to a valid argument and vice versa:
\begin{quote}
    An argument is logically valid just when conditional [sic.], with the conjunction of the premises as antecendent and the conclusion as consequent, is logically true. We can readily slip from one notion to the other[...] \cite[p.~13]{beallrestall}
\end{quote}
This is undoubtedly true for standard logic. However, it is false for some non-standard logical theories. For instance, in dialetheist logical systems like BXTT (see for example \citeA[p.~31-32]{spandrels}) and Priests LP with $\rightarrow$ and semantics $\Delta$ (see \citeA[Chapters~5~and~6]{incontradiction}) it is not always possible to `slip' from a valid argument to a logically true conditional. Even though $\alpha,\alpha\rightarrow\beta\vdash \beta$ (\textit{modus ponens}) is a valid inference rule in these systems, the formula \mbox{$(\alpha\wedge(\alpha\rightarrow\beta))\rightarrow\beta$} (`\textit{pseudo modus-ponens}') is not a theorem. If these systems would include \textit{pseudo modus-ponens} as a theorem, a Curry paradox could be constructed and the systems would be trivial (see for example \citeA[p.~83]{incontradiction}).
%In fact, a logical system with a transparent truth predicate, substitution of equivalent formulas, a self referring formula \mbox{$\delta=T\underline{\delta}\rightarrow\beta$} for any $\beta$ and \textit{modus ponens} cannot include the formula \mbox{$(\alpha\wedge(\alpha\rightarrow\beta))\rightarrow\beta$} (`pseudo modus-ponens') as a theorem without being trivial. In a consistent logical system that fulfils these conditions, the shift from the valid inference $\alpha\rightarrow\beta,\alpha\vdash\beta$ (modus ponens) to the formula $((\alpha\rightarrow\beta)\wedge\alpha)\rightarrow\beta$ must therefore be impossible. Otherwise, a Curry paradox can be constructed, see for example \citeA[p.~83]{incontradiction}.

Furthermore, from a modern perspective we can say that the notion of validity is more fundamental than the notion of logical truth. To specify the valid arguments in a logical system, a consequence relation $\vdash$ is defined, such that the valid arguments are exactly those, for which the consequence relation holds. More specifically, the consequence relation is a relation between the premises and the conclusion of an argument and it holds exactly in those arguments in which the premises establish the truth of the conclusion. A logical truth is then a sentence $\alpha$, for which $\emptyset\vdash\alpha$ holds, i.e. that occurs as a conclusion in an argument without premises. Formally, this could mean that these sentences can be derived from axioms only, proven in a calculus of natural deduction or that they are valid in every model. Informally, a logical truth is a sentence which one does not have to argue for (because it is true `by itself`). Its truth does not have to be established by additional premises. But certainly not just any relation can count as a consequence relation for it should hold only for \textit{valid} arguments. But when do the premises actually establish the truth of the consequence? Or in other words: what exactly does it mean for an argument to be valid?

%I will adopt the definition of validity proposed by Beall and Restall. They formulate the following principle (the `Generalised Tarski Principle', GTP) to define what they hold to be a valid argument:

There is a definition of validity proposed by Beall and Restall. They formulate the following principle (the `Generalised Tarski Principle', GTP) to define what they hold to be a valid argument:
\begin{quote}
An argument is valid$_x$ if and only if, in every case$_x$ in which the premises are true, so is the conclusion. \cite[p.~29]{beallrestall}
\end{quote}

%Problem: funktioniert nicht für proof theoretic semantics! dort ist nicht von wahrheit die rede. Moegliche loesung: `true$_x$' einfuehren. Dann kann z.b. definieret werden: argument ist valid gdw. in jedem fall, wenn die praemissen beweisbar$_x$ sind, ist auch die konklusion beweisbar$_x$.

The index $x$ indicates that there may be different kinds of validity and that for each kind only a certain class of cases has to be considered. Thereby, the principle guarantees that the consequence relation is truth-preserving in all the relevant cases and leaves open the question of which cases are the relevant ones. Thus, the GTP leaves room for logical pluralism. %I will come back to the GTP later.
However, there is a problem with this definition of validity: it rules out theories of validity that do not involve the notion of truth, such as proof-theoretic validity, where validity is not seen to be a semantic notion, but one attributed to proofs. Simply put, a proof-theoretic account of validity would say that an argument is valid if and only if the conclusion can be derived from the premises. Therefore, Beall and Restalls GTP has to be modified in such a way that it also accounts for proof-theoretic theories of validity. I suggest to introduce two new variables:

\begin{quote}
\textbf{Generalised Generalised Tarski Principle (GGTP)} An argument is valid$_x$ if and only if, in every case$_x$ in which the premises have property$1_x$, the conclusion has property$2_x$. \label{principle:GGTP}
\end{quote}

A truth-semantic theorist would define cases$_x$ to be different models or possible states of affairs and fill in for both, property$1_x$ and property$2_x$, the property of a statement to be true, or accurately depict a state of affairs. A proof-theoretic definition of validity in form of the GGTP would be to define an argument to be valid if and only if, whenever in a certain logical calculus (case$_x$), the premises are assumed (property$1_x$), the conclusion can be derived (property$2_x$). From now on I will take the GGTP to be the scheme every definition of validity has to fit in.

The second point that might be disputed is that a logical theory is normative. A logical theory might for example be seen as a theory that simply makes explicit the logical structure of natural language or that shows us how people normally argue. However, if logical consequence is a relation between premises and a conclusion and logic is concerned with logical consequence, logic is concerned with arguments. More precisely, logic is concerned with the validity of arguments, for an argument is valid if and only if the consequence relation between the premises and the conclusion holds and the question whether this relation holds for a certain set of premises and a conclusion is the question which we try to answer in logic. However, arguments are not abstract objects of any kind. They are something we can find in everyday live - we argue to defend our views, to convince others, to prove mathematical theorems or to check whether certain beliefs contradict each other. Arguments are something we make and by telling us which arguments are valid and which are not, logic tells us which arguments we \textit{should} or \textit{can} make and which we should not make. So, by judging arguments, a logical theory also judges our practice of arguing and has, therefore, a \textit{normative} function.

Later on, I will say more about what a logical theory is, how the consequence relation fits in and how the GGTP can be understood technically. Before I do this, I will turn to the question of how there can be reasonable dispute concerning logical consequence.

\section{An application of RE in the philosophy of logic}
In this section, I will work out the central problem that motivates the application of the RE-theory in the philosophy of logic. I will further discuss, why the RE-theory is a good candidate to answer the question of how people can rationally disagree on the notion of logic and how I think, RE can be used to show that a logical theory is a rational theory.

\subsection{Debating on logic}\label{sec:debating}
Philosophical debate involves making and criticising arguments. As I argued above, the discipline that investigates arguments and tries to answer the question which arguments should be accepted as valid, is logic. When we do logic, we try to spell out a formal theory to investigate whether a certain argument is valid or not. So one could think, to have a philosophical debate, we must \textit{presuppose} a logical theory, otherwise we would not know the `rules' of arguing and \textit{a forteriori} the rules of philosophising. One could further suggest that it is therefore utterly impossible to have a successful philosophical debate on the question of which logical theory we should use. If two philosophers were in disagreement on the question of which logical system is preferable, it might be expected that each philosopher argues according to the theory she prefers. But then neither will accept the arguments of her opponent and their debate would be pointless.

Obviously, this view is naive. There clearly is a fruitful philosophical debate on logical theories. Even fundamental principles like the \textit{law of non-contradiction (LNC)} and the \textit{law of the excluded middle (LEM)} are frequently discussed, as, for example, the debates around dialetheism and intuitionism show. Philosophers use arguments to establish their views of logic and they consider it possible to convince others who adhere to rival theories. Furthermore, the participants in such debates do not seem to think that their opponents hold irrational views and, even though they argue in favour of different logical theories, they seem to have reasonable discussions. By the principle of charity we should assume therefore that there are different rational views in the philosophical debate around logic. Furthermore, that there is a philosophical debate on these different views implies, I think, that there must be some common ground. This includes mainly three points.

First, there are some views in logic that are widely shared among the participants of the debate. For example, inferences like \textit{modus ponens} and the introduction of the existence quantifier ($\phi[a]\vdash\exists x\phi[x]$) are normally accepted as valid. There may therefore be a significant overlap of accepted arguments in a philosophical debate. As long as philosophers only use argument forms that are commonly accepted as valid, there will be no problem. But it seems like this kind of common ground can only explain why advocates of different logical theories can \textit{sometimes} have a philosophical debate. However, the debate concerns exactly those principles and inferences that are \textit{not} broadly accepted, and there are coherent logical theories that differ in respect of the argument forms they define to be valid. This shows that, in some cases, the acceptance of one inference does not strictly lead to the acceptance of another. So, the question remains: how can people who differ in their understanding of valid arguments, have a fruitful philosophical discussion of the inferences and principles that are not commonly accepted among them?

Second, participants of a philosophical debate on logic have to share some concepts. For example, they need to have a shared notion of logic. There should be agreement on what a logical theory is, and what it should accomplish. In the previous section, I argued for a view that defines a logical theory as a formal system that tells us which arguments we should accept as valid, by defining a consequence relation which, in turn, is characterised by the GGTP. For this definition to be commonly accepted, there also has to be agreement on the defining terms like `formal system' and `argument'. Furthermore, a logical theory must be formulated in a way that it is comprehensible for others i.e. that others can understand how it is used and how, for example, theorems can be proved or how its semantics works. But how can we be sure that the different approaches to logical consequence do not `pour' into the language itself, making it impossible, for example, for one person to understand the proofs another gives in her logical system? After all, what a proof is, depends on the definition of logical consequence.

In a Wittgensteinean fashion, I would argue that a formal system essentially is a set of rules, and that all that is needed to understand that system (and its proofs) is the ability to follow the rules it defines. However, rule following is something every competent speaker of a natural language is able to do. There is therefore no problem for two adherents of different logical theories to understand the view of logic the other person holds. All they need to do is to make explicit the rules according to which they draw inferences. 

But even if this is not found to be convincing, I do not think that there is a serious problem here. In general, people are able to understand different logical systems. For example, they do not disagree whether or not a intuitionist is committed to accept a certain argument or whether a dialetheist, who thinks that Bealls system BXTT is an adequate logical system, is committed to accept a certain logical principle. And if a dialetheist logical system can prove a contradiction due to a transparent truth predicate, everyone would agree that the standard logician would be missing the point if she would infer from this fact, that the system is trivial. All this shows, I think, that it is very well possible for an advocate of a certain logical theories to understand how her opponents in the debate argue. There seem to be in fact no problems of understanding of the kind described above. A problem only arises, if someone does not specify the rules of inference she thinks to be the right ones at all. There would be no way to check whether such a view is coherent -- the person who brought forward this `theory' would classify some arguments as valid and some as invalid without referring to an underlying principle. She could even change her mind on the validity of a certain argument over time and claim that the theory did not change. It would be impossible to use or even understand such a `theory'. Such a view must certainly be rejected as irrational. It would not even be a logical theory in the sense defined here: a logical theory specifies a logical consequence relation and this must, according to the GGTP, be done by specifying the variable `cases$_x$' in which a specific property$1_x$ of the premises of a valid argument must guarantee the conclusion to have a specific property$2_x$. By not specifying any rules of making valid inferences, and thereby, a forteriori, not defining what it means for an argument to be valid, this requirement is clearly not met.

This leads us to the third and last point: there must be some common criteria of what a \textit{rational} position in the discussion about logical consequence is. As argued above, these criteria should leave room for different logical theories, but there certainly are logical theories that are irrational. As already mentioned, a logical theory should specify a set of rules of inference. A logical theory cannot count as a rational theory if it is not clear how it actually defines logical consequence. Another standard example of an irrational logical theory is one that leads to trivialism, the position that every sentence is true. A trivial logical theory would yield every argument valid, which must certainly count as irrational.\footnote{An extensive discussion of trivialism and why it is an irrational view can be found in chapter 3 of \citeA{priest2}.}

The picture of philosophy of logic I gave so far, suggests a form of \textit{logical pluralism}. Logical pluralism is defined by \fullciteauthor{beallrestall} to be the claim that there is more than one admissible account of logical consequence and that `the pluralist endorses at least \textit{two} instances' of the GTP \cite[p.~29]{beallrestall}. The pluralism I argued for, is more moderate. It does not `endorse' more than one account of logical consequence. It does not even say that is not \textit{rational} to endorse more than one account of logical consequence, but only supposes that more than one account is \textit{rationally acceptable} or \textit{justified} (I will use these two terms synonymously in this essay), i.e. that reasonable disagreement on logical consequence is possible. However, as the considerations in this section showed (and further considerations in the next section will confirm), even this moderate form of logical pluralism requires some further explanation. In the rest of this essay, I will argue that the theory of RE can provide an explanation of this pluralism.

\subsection{How to justify a logical theory?}
The idea of applying the method of RE to justify a logical theory goes back to the very inventor of RE \cite{sep-reflective-equilibrium}, Nelson Goodman:
\begin{quote}\label{quote:goodman}
    How do we justify a \textit{de}duction? Plainly, by showing that it conforms to the general rules of deductive inference. [...] But how is the validity of rules to be determined? [...] Principles of deductive inference are justified by their conformity with accepted deductive practice. Their validity depends upon accordance with the particular deductive inferences we actually make and sanction. If a rule yields inacceptable inferences, we drop it as invalid. Justification of general rules thus derives from judgements rejecting or accepting particular deductive inferences. [...] The point is that rules and particular inferences alike are justified by being brought into agreement with each other. [...] The process of justification is the delicate one of making mutual adjustments between rules and accepted inferences[.] \cite[p. 63-64]{Goodman}
\end{quote}

The idea here is that we start with some arguments or argument forms (what Goodman calls `deductive inferences') of which some are accepted as valid, and some are hold to be invalid. Furthermore, we need a logical theory to start with. The goal is then, to bring the theory and our convictions concerning the validity of arguments into accordance (in the following, I will refer to these convictions also as \textit{commitments}). This means in the best case that, according to our logical theory, all the arguments we accept are valid and none of the arguments we reject are valid. I will simply call such a situation an \textit{RE}. However, establishing an RE is not done by simply stipulating a theory that fits our needs. We reach an RE by `making mutual adjustments between the rules and accepted inferences', as Goodman writes (see quotation above). The process of mutually adjusting the theory to the commitments and the commitments to the theory respectively, is usually described as working back and forth between the theory and the commitments which our theory should account for \cite{sep-reflective-equilibrium}. After each adjustment, we reach a new \textit{RE-stage}. I will refer to the process of going from stage to stage until an RE is achieved as an \textit{RE-process}. For a quick illustration, let us suppose we start by checking, if our theory yields unaccepted arguments as valid. If it does, we change something in the theory and reach the next RE-stage. In the next step, we review our commitments by looking for an argument that is considered to be valid but is invalid according to our theory (or vice versa). If there is such an argument, we might move to the next RE-stage by changing our mind on it and rejecting it as invalid or, respectively, if our theory tells so, by accepting it as valid. After that, we turn back to our theory and try to adjust it, so that it fits our commitments better, before reviewing our commitments again. This process continues until we reach a logical theory that accounts exactly for our commitments.

\subsection{How RE can establish justification}
Let us sum up Goodmans suggestion. We show that the validity of an argument is rationally acceptable by showing that it is valid according to the `general rules of deductive inference' \cite[p. 63-64]{Goodman} and these rules are again rationally acceptable if they are the result of an RE-process. But there is, as mentioned above, not one unique set of rules of deductive inference that is commonly accepted to be the best one but several rival approaches to logical consequence. With the theory of RE we are now able to explain this: Goodmans characterisation of the RE-process does not guarantee that everyone carrying out such a process ends up with the same theory. Arguably, there may be a big overlap concerning the initial commitments of different RE-processes (for example, simple argument forms like modus ponens will normally be accepted while arguments that are normally seen to be fallacies will initially be held as invalid). Nonetheless, the process may vary significantly in respect of the adjustments made in each step. As a consequence, different RE-processes may have different outcomes.

So if we take Goodmans proposal seriously, we can explain how philosophers can disagree rationally -- a logical theory can be held rationally if it is the result of an RE-process and an RE-process can result in different logical theories. Therefore, different logical theories can be held rationally (i.e. these theories are justified). Consequently, one might conclude, the question `Is it rational to accept the logical theory $\mathcal{L}$?' is equivalent to the question `Was the logical theory $\mathcal{L}$ developed in an RE-process?'. But this is misguided for two reasons: first, what is proposed here is not the claim, that a rational logical theory \textit{necessarily} is the outcome of an RE-process but that being the outcome of an RE-process is \textit{sufficient} for a theory to be justified. Second, it is not about how a theory that was \textit{developed}. This is obvious, for to decide whether or not a logical theory is rational, it should not matter how the theory \textit{came into existence}. It can, of course, be rational to accept a logical theory without knowing how it was developed. The question is \textit{for what reasons} the theory is accepted. The suggestion given here is, that it is rational to believe a logical theory that \textit{demonstrably can be the outcome} of an RE-process. 

To sum up, the correct picture, I think, is this: a logical theory $\mathcal{L}$ can be justified by developing an RE-process that ends with an RE in which the theory is exactly the logical theory $\mathcal{L}$. In other words, S rationally accepts a logical theory, if S knows that an RE-process can be carried that results in this theory. A certain argumentative practice then is justified, if there is a theory justified in this way and this theory accounts for the argumentative practice in question. But now another question arises: how can such a procedure be a justification of a logical theory? To answer this question, we cannot use a theory of justification that relies on the notion of inference. This becomes clear when we try to spell out such an account in the case of the justification of a logical theory. 

A proposition is often said to be believed rationally if it was inferred from a set of propositions that are justified themselves, i.e. if the former is a consequence of the latter. The same cannot be true in general for beliefs concerning the consequence relation, for to \textit{infer} a proposition it must already be clear \textit{how} to infer anything. But what a valid inference is, is defined by a consequence relation itself. Surely this could not just be any consequence relation, for the question of whether a certain belief is held rationally -- and therefore the question of whether or not a belief is inferred correctly -- must have an answer with intersubjective validity. To demand that every belief about the consequence relation has to be inferred from another belief would therefore presuppose one distinct notion of logical consequence. However, such an a priori logical consequence does not fit with the account of moderate logical pluralism specified above. If there were a unique a priori notion of logical consequence, there could only be disagreement on logical consequence if a person would be wrong about the a priori concept of logical consequence. 

Again, there may be an intersubjective practice of inference here, i.e. some inferences that are widely accepted to be justifying a belief. However, the case which we are interested in, namely the philosophical debate on logic, is a case in which we ask for the acceptability of inferences that are not intersubjectively accepted as valid and so not intersubjectively accepted as justifying. Therefore, there is no intersubjectively accepted practice of inferential justification.

It may be argued that there is such an a priori -- or at least intersubjective -- notion of logical consequence, but that this notion is not `rich' enough to specify an exact formal definition of the logical consequence relation. But this leads to another problem, for we now have to explain from which propositions the principles of logical consequence are inferred. To claim that there is a `weak' a priori notion of inference and that all beliefs about logical consequence must fit into an inferential system of beliefs leaves us only two options.

The first option is to claim that, by justifying each belief concerning logical consequence, we eventually reach some \textit{fundamental principles} that do not have to be backed by more fundamental propositions. This, however, brings us back to the problem we mentioned before: if there are such fundamental principles and beliefs are held rationally only if they were inferred from these fundamental principles, reasonable disagreement on the notion of logical consequence would not be possible and moderate logical pluralism would be false. 

The second option is to demand for a logical theory that it only has to be \textit{coherent} according to the `weak' a priori notion of inference. But this will not work either, for coherence alone cannot guarantee for rationality. Even if a logical theory looks strikingly irrational, it might still be coherent according to its own principles. For example, as \citeA{haack} shows, if we accept the inference rule she calls \textit{modus morons} (`affirming the consequent') as an inference rule in a logical system, this principle looks about as plausible as \textit{modus ponens} from an internal perspective.

Neither foundationalism nor coherentism can therefore account for our moderate form of logical pluralism. The situation is more complex. The reasons for a logician to reject the validity of one class of arguments and to accept the validity of another are manifold, and may be pragmatic, systematic or philosophical to name only a few. But surely, a logician must have good reasons to favour a certain logical theory -- not every theory that looks like a logical theory can count as a rational logical theory.

Even if there are no a priori principles of logical consequence which strictly lead us to a complete logical theory, there seems to be a core notion of logical consequence that can be expressed in form of the GGTP. To describe a logical theory then is to specify what the variables `cases$_x$', `property$1_x$' and `property$2_x$' in the above formulation of the GGTP mean (see section \ref{principle:GGTP}, page \pageref{principle:GGTP}). Working out a systematic characterisation of the cases in which truth is preserved might be done in different ways. This is, as suggested above, the reason for disagreement on logical consequence. One philosopher might reject the LNC in turn of fleshing out a logical theory, another might stick to it. If their disagreement on the resulting logical theories is reasonable, either must be justified to hold her view on logical consequence. But then both must have acted rationally by rejecting and holding to the LNC, respectively. This means that the `boundaries' of rationality must allow for either of these steps. Suppose conversely that both did work out their logical theories in a process that was lead by principles of rationality. Then both are justified in holding their views on logical consequence, even if they disagree on them. In other words, if there is a process in which each step is governed by rules of rationality and that leads to the logical theory $\mathcal{L}$, then it is rational to accept $\mathcal{L}$. If these rules do not determine the exact outcome of the process, different logical theories can be justified and therefore held rationally. The claim here is, of course, that an RE-process can be seen as such a process, and that because of this, a rational logical theory can be justified by carrying out an RE-process.

To show how an RE-process can serve as a justification of certain views on logical consequence, let me illustrate my point. Imagine a person that is learning about the standard propositional calculus. Suppose further that the first thing this person learns is that, according to the propositional calculus, the following argument is valid:

\begin{center}
\begin{tabular}{c p{9cm}}
P1 & If I am king of France, I'm not king of France.\\\hline
C & If I am king of France, God exists. 
\end{tabular}
\end{center}

She would most likely reject the whole argument as nonsense and claim that the premise does not establish the truth of the consequence. Furthermore, she would not be satisfied by hearing that for the first premise to be true, the proposition `I am king of France' has to be false and therefore the conclusion is true because its antecedent is false. It might seem to the person hearing all this the first time that it is not very rational to embrace a theory with such results. She would stress that nobody would argue like this and she would be surprised that a lot of philosophers thought and still think that the propositional calculus is an appropriate tool to investigate and criticise arguments. How can we convince her, that the propositional calculus actually \textit{is} a good and rational theory of reasoning?

Obviously, we cannot simply lay out the whole formalism and prove that the theory really yields the counterintuitive arguments valid, for this would not be taking serious the concerns of the person questioning the rationality of the the formal system itself. What we must provide is `the story behind' the formal system. We might do this by starting with some examples of arguments we suppose to be valid, and, to convince our sceptic, initially assume that the seemingly paradoxical arguments in question are actually false (we will only consider arguments that can be analysed with propositional logic for this is the domain of the theory in question). Next we might try to give a systematic account of a consequence relation according to which the arguments initially chosen to be valid are valid and the others are invalid. This will turn out to be not that easy and we might have to introduce some exceptions for our theory to account for our position concerning the validity of the arguments we chose at the beginning. However, for we do not want to end up with an unsystematic patchwork, we might be forced to make some trade-offs in building our theory and as a result reconsider our initial position. This means, systematicity might force us to change our view on some of these arguments. But then again, we might hold on to some of our initial judgements and look for systematic improvements of our logical theory to make it account better for our worked-over set of valid and invalid arguments. It is important that we can motivate each step in this process with principles of rationality that we share with our critic. If the propositional calculus is a rational theory according to our shared notion of rationality, we will supposedly end up with the very logical theory of propositional calculus and with accepting exactly the arguments that are valid according to standard sentential logic. Of course, this is just an instance of an RE-process as it was described above. 

The person questioning the rationality of our theory does not have to agree with us on the notion of logical consequence after following the RE-process we spelled out. The goal is not to persuade her to believe that this is the correct account of logical consequence. What we want is to show that our account is a rational one and we try to do this by demonstrating that there is a rational development process that leads up to the theory of our choice. If the other person agrees that each step in the RE-process was done rationally, she will have to admit that we act rationally by holding on to our theory. %There might be some special cases and I will return to this point later. 

The sceptic in this example illustrates two things. Firstly, is not hard to judge a logical theory by simply examining its consequences. To decide whether a certain account of logical consequence is rational, the exact reasons for a person to accept this account must be considered. There may be a lot of different kinds of reasons and they can be made explicit by spelling out an RE-process. 

Secondly, the notion of rationality may be relative. Justifying a logical theory is done by showing that the process was rational \textit{according to some shared principles of rationality}. This does not imply relativism about rationality. There may well be some objective standards of rationality but this does not mean that everyone knows and accepts them. An RE-process may therefore justify a theory only relative to certain standards of rationality. The background assumption to this is, of course, that there are no objective standards of rationality \textit{that would exactly determine} the outcome of a rational RE-process. As argued above, it seems unlikely that there are such standards. However, I think there are some fundamental principles that must be considered to develop a formal logic. I will discuss this in the next section, where I will develop a more precise account of RE, based on the considerations in the last sections.

\section{Spelling out RE}\label{sec:spelling_out}
After discussing the central terms and concepts, I will now try to give a more precise characterisation of an RE-process and its constituents. I will first define central technical terms and then discuss the principles that govern a rational RE-process. The main thesis of this section is that there are important guidelines for an RE-process and that these guidelines have to be considered (and normally are) by all participants of the philosophical debate on the notion of logic.

\subsection{The formalities}
The goal of the rather formal characterisations I will now give is not to work out a formal model to compute an RE-process (like the one put forward by \citeA{re-model}) or something the like, but rather to give a general, neutral, and exact specification of the terms and concepts mentioned so far. I do this in order to show that an RE-process can be specified in a way that does not presuppose central logical principles. This is, of course, crucial, for the characterisation of the RE-process and the definition of a logical theory must be independent of specific logical principles -- it should after all be possible to show the rationality of \textit{different} logical theories in an RE-process. We should, for example, not somehow put the LEM or the LNC into our definition of the consequence relation, and thereby rule out a lot of logical theories, if we think that it can be rational to reject these principles in an RE-process.

I will nevertheless make use of a logical concept to give this characterisation, namely I will make use of sets, what may look problematic. However, I think that this concern is misplaced. Firstly, I will not make use of powerful features of set theory. I will only presume that sense can be made of terms like `subset' or `being an element of a set'. Secondly, if there would be a logical position that rejects this basic notion of a set, the following characterisation might still be understood as a characterisation of an RE-process \textit{in classical terms}. These terms may be criticised as being not appropriate, or as being problematic in some sense but this does not mean that they can not been understood at all. Finally, even if the concept of a set would bring a lot of `logical import', this would not mean that this logic would run into the RE-process. The process is just \textit{described} by means of sets. Nothing will logically be inferred, for example, from the fact that a language is depicted as a set of formulas. 

Let us start with the formalities. At each RE-stage, there are some commitments and a logical system. An RE-stage is therefore taken to be a tuple of the form $(\mathcal{C},\mathcal{L})$ where $\mathcal{C}$ is the set of commitments and $\mathcal{L}$ is a logical theory. There are `positive' and `negative' commitments, i.e. some arguments we consider to be valid and some arguments we consider to be invalid. Furthermore, there may be arguments for which we do not want to decide initially whether they are valid or not. Such `uncategorised' arguments may become accepted as soon as the logical theory yields them valid. They do not influence the process as long as they are not accepted or rejected but as soon as they are, they have to be taken into account when adjusting the theory.

So, a set of commitments contains three elements, the set $\mathcal{C}^+$ of arguments accepted as valid, the set $\mathcal{C}^-$ of arguments considered as invalid and the set of uncategorised arguments $\mathcal{C}^?$.% This description of an RE-stage is pretty obvious and, I think, uncontroversial.

An argument consists of a set of premises and a conclusion. This is quite uncontroversial. However, as \citeauthor{beallrestall} notice, there are reasons to allow for an argument to have multiple conclusions \cite[p.~14]{beallrestall}. I cannot go into details of this question here. Since arguments formulated in non-proof-theoretic disciplines normally seem to have only one conclusion, I take this to be the more `natural' approach. 

It is supposed here that arguments are adequately formalised, that is, they are adequately translated into the language of a formal logical theory $\mathcal{L}$ (I will say more on this later on). A logical theory is depicted by the set of well formed formulas $L$ and a consequence relation $\vdash_\mathcal{L}$. Because an argument consists of some premises and a conclusion, where the conclusion is a \textit{consequence} of the premises, the consequence relation is a relation $\vdash$ between a set of sentences of $L$ on the left side (the premises) and one single sentence on the right side (the consequence). Consequently, an argument is a tupel $(\Gamma,\alpha$), where $\Gamma$ is the set of premises and $\alpha$ is the conclusion.

It can be assumed that the same argument cannot be accepted and not accepted at the same time, in formal terms that $\mathcal{C}^+$, $\mathcal{C}^-$ and $\mathcal{C}^?$ are pairwise disjoint. It is, I think, very plausible to postulate this, for it is simply not possible to accept and not accept the validity of an argument at the same time. To reject the validity of an argument $(\Gamma,\alpha)$ in an RE-process would mean to go from a stage $(\mathcal{C}_i,\mathcal{L}_i)$ to a stage $(\mathcal{C}_{i+1},\mathcal{L}_i)$ where $(\Gamma,\alpha)\in\mathcal{C}_i^+$, $(\Gamma,\alpha)\notin\mathcal{C}_i^-$, $(\Gamma,\alpha)\in\mathcal{C}_{i+1}^-$ and $(\Gamma,\alpha)\notin\mathcal{C}_{i+1}^+$. % It is possible to reject the validity of an argument without including the argument in the set of arguments considered as invalid. This means that the logical system at a later stage does not have to account for this (kind of) argument anymore.

\subsection{Principles of rationality}
So the goal of carrying out an RE-process is to develop a consequence relation. This goal is achieved when at some RE-stage our logical theory applies for every argument we accept and does not apply for any argument we do not accept. However, such a situation cannot just be established by any means. The goal is, as discussed in the last section, to specify a \textit{rational} procedure that leads to a logical theory. Such a procedure is, as suggested, a procedure in which every step is done rationally, so there have to be certain goals and principles guiding an RE-process besides bringing the commitments and the theory into accordance. In the following, I will discuss four principles that, as I will argue, are fundamental principles of a rational procedure to justify a logical theory. This is not supposed to be an extensive list but a list of \textit{minimal} standards of such a procedure. Similar principles have already been suggested, for example by \citeA{brun2} and \fullciteA{re-model}.

\paragraph{Systematicity} To define the consequence relation by specifying it for each argument separately would be a very unsystematic procedure. Normally a logical theory includes a axiomatic or semantic characterisation of the consequence relation (or both). For systematicity is clearly a goal in an RE-process, it can be postulated that it should be possible for the agent carrying out the process to give a systematic definition of the consequence relation. Of course systematicity comes in degrees and is a rather vague notion, but it will be non-controversial that a theory defined by syntactic calculus is more systematic than some patchwork definition of a consequence relation with a lot of exceptions. In every case, the rules of inference have to be explicitly specified, as argued in section \ref{sec:debating} on page \pageref{sec:debating}. A logical theory that does not define the rules of inference is not a rational theory -- it can hardly count as a logical theory at all. The criterion of systematicity also applies to the language of the logical theory -- there has to be a specification of the vocabulary and a systematic definition of the formulas. 

\paragraph{No change of topic} A logical theory, in the sense it is understood here, includes a formal system that is concerned with the validity of arguments in natural language. There may be different logical theories reasonably held but if a theory is not about the validity and invalidity of arguments in natural language, it is not a logical theory. It should therefore be avoided to change the object of investigation during the RE-process. To guarantee that this does not happen, it can be postulated that, at the beginning of the RE-process, a translation scheme must be given which explains how the formulas and arguments of the formal language can be interpreted as expressions and arguments in natural language. This also fixes the kind of arguments that the resulting logic should account for. For example standard logic is not concerned with sentences including (non-eliminable) truth predicates. It may therefore be the outcome of an RE-process that starts with a translation scheme for a language without truth predicate. %does not say that the inference from the sentence `This sentence is false' to its truth/falsity is not \textit{valid} -- it 

\paragraph{Applicability and usefulness} A logical theory must be applicable and usable in some way. This means on one hand that it must be possible to interpret its language which is guaranteed by the translation scheme mentioned above. On the other hand a logical theory should also be of good use in some way. This could mean that it is an appropriate tool for mathematical reasoning, that it can be used to study arguments and fallacies made in everyday life or even that it can be used to make sense of religious texts (as \citeA{thewayof} suggest). An extreme case of a useless logic would be a trivial logic that yields every argument valid. A logical theory can of course be of more or less use and it can apply to a big part or only to a small fragment of natural language. The goal of applicability and usefulness is gradual and can be met more ore less.

\paragraph{Reasonable steps} Each step in a process that changes the theory should lead to a theory that meets the standards of rationality (at least to a sufficient degree) and should be done only if it makes the theory better according to these standards or makes the theory account better for the commitments. On the other hand, commitments should only be changed if the theory requires to do so. This means, for example, that a certain modification in the definition of the consequence relation should only be done if, for example, it leads to a definition that is more systematic. Systematicity may be achieved by reducing exceptions in the definition of the consequence relation. Furthermore, there should not be big steps. The goal of an RE-process is to make \textit{transparent} the considerations that lead up to a certain logical theory and this is only possible if these considerations are shown in detail. I therefore think that it can be demanded that, at each step, only one commitment is considered. We take one commitment that we are ready to reject, and test it against our theory, or we take one commitment, and check whether our theory accounts for it. This also seems to be what Goodman suggests when he writes:

\begin{quote}
    A rule is amended if it yields an inference we are unwilling to accept; an inference is rejected if it violates a rule we are unwilling to amend. \cite[p.~64]{Goodman}
\end{quote}

%A further cirterion advocated in \citeA{brun2} as well as in \fullciteA{re-model} is the criterion called `faithfulness'. It requires that the commitments at each RE-stage reflect the initial commitments to a certain degree \fullciteA[p.~7]{re-model}. \fullciteauthor{re-model} argue that this principle prevents the change and that it is required due to the (supposed) initial credibility of the initial commitments. However, the change of topic is prevented by a separate principle and the emphasis put on the credibility of the initial commitments

What does all this mean for an actual RE-process? As argued above, to respect the principle \textbf{No change of topic}, a translation scheme must be given and the arguments considered as valid and invalid respectively must be translated according to this scheme at the beginning of the process. The difficulties in establishing such a translation scheme must be set aside for now. In the following I will just assume that it is possible for any adequate formal language to spell out a translation scheme which enables us to give a formalisation of arguments in natural language. Furthermore, to guarantee a minimum degree of \textbf{Systematicity}, a consequence relation should have a proper definition. A definition can be given in terms of inference rules, models, an axiomatic calculus or the like. It may even include exceptions, but it should specify the precise conditions in which a given well formed formula $\alpha$ $\mathcal{L}$-follows from a set of well formed formulas $\Gamma$ i.e. whether $\Gamma\vdash_{\mathcal{L}}\alpha$ holds or not. This leads to the formulation of the following principle:

\spec{Proper definition of `$\vdash$':}{}{For any RE-stage $(C,\mathcal{L})$, a proper definition of $\vdash_\mathcal{L}$ must be given that determines for every $\alpha\in L$ and every $\Gamma\subseteq L$ whether $\Gamma\vdash\alpha$ holds or not.}

This principle requires the agent who carries out the RE-process to specify the cases (cases$_x$) in which a certain property (property$1_x$) of the premises implies that the conclusion has a certain property (property$2_x$). This can be done by giving a semantic theory, i.e. by defining an interpretation or a model of well formed formulas and by requiring for an argument, to be valid, that every model of the premises is also a model of the conclusion. It can also be done by giving axioms and inference rules, such that an argument is valid if and only if the conclusion can be proven from the premises. The crucial point is that the definition of validity unambiguously determines which of the considered arguments are valid and which are not. Notice that, when this principle is met, we have a definition of validity that can be brought into the form of the GGTP, for an argument $(\Gamma,\alpha)$ is said to be valid if and only if $\Gamma\vdash\alpha$ holds. %So, if a definition of the consequence relation determines for each argument whether or not it is valid, we know from this definition the cases that have to be considered and the properties that must hold for the premises and the conclusion in these cases for an argument to be valid.

Notice that the \textbf{proper definition of `$\vdash$'} does not require that there is an \textit{(efficient) procedure to decide} whether or not the consequence relation holds for a given tupel $(\Gamma,\alpha)$. There is, for example, a proper definition of first order predicate logic entailment but there is no procedure to decide the set of first order tautologies. It is also not required that for a language $L$ with a negation-operator any direction of the biconditional `$\Gamma\nvdash\alpha\Leftrightarrow\Gamma\vdash\neg\alpha$' holds. There may, for example, be an $L$-formula $\alpha$ with $\Gamma\nvdash\alpha$ and $\Gamma\nvdash\neg\alpha$

A proper definition of `$\vdash$' must be provided at any RE-stage and therefore a change of the consequence relation will mainly consist in revising the definition of the consequence relation. In practice this could mean that an inference rule or an axiom is rejected or that a further restriction to the models is introduced.

One might object that this account of consequence smuggles the principle of non contradiction into the meta-language by assuming that for any formula $\alpha$ and any set of formulas $\Gamma$, $\Gamma\vdash\alpha$ holds or does not hold, but not both. A dialetheist, it might be argued, could accept that a sentence can follow and not follow from a set of premises at the same time, and might therefore reject my account. I do not think that this is correct. If I am right to suppose that logic is about arguing, and that it is normative by telling us which arguments we should accept and which we should not accept, a logical theory that states that $\Gamma\vdash\alpha$ holds and does not hold at the same time would require us to accept and not accept the corresponding argument in natural language. This, however, is not possible, as even Priest admits \cite[p.~618]{priest1}. But if something cannot be done, one cannot be obligated to do it (this is an application of the principle `ought implies can'). Such a theory must therefore be false.

Philosophical background assumptions may play a role in an RE-process in two ways. First, the initial commitments will define the scope of the logical theory that will be developed. For example, a truth predicate or modal operators can be included in the language. Second, whether a certain commitment is rejected lightly or, not may be motivated by philosophical reasons. An intuitionist will most likely, when carrying out an RE-process, initially reject the argument $(\emptyset,\phi\vee\neg\phi)$ and will not consider to accept it in turn of the process, for she has philosophical reasons to do so. %Furthermore, she may initially accept an argument like $(\{\neg\neg\phi\},\phi$ but she will then have to consider to reject this argument in the course of the process to make the theory account for the commitments.

It may be argued that, to judge the rationality of a logical theory, the decisions to hold on to a certain commitment and to readily reject some other have to be rational too. This is correct, but to give standards of rationality for philosohpical views seems to be very hard (if not downright impossible). However, I think that in the most cases, the principle of charity demands that we do not right away reject the philosophical views of a person as utterly irrational, even though we may not agree on them. We may of course question the rationality of this persons views on logical consequence if she is not able to explain these views in an RE-process.

\subsection{What an RE-process could look like}
Now a clearer picture of an RE-process can be drawn. To start an RE-process, initial commitments must be specified by defining $C_1^+$ and $C_1^-$. There may also be some arguments in $C^?$ for which we do not want to decide yet, whether we hold them to be valid or not. In the course of the RE-process, we may reconsider them and move them into $C^-$ or $C^+$, depending on the logical theory developed so far. This means that some arguments or  argument types are accepted as valid or invalid. Furthermore, an initial systematisation in form of a logical theory $\mathcal{L}_1=(\vdash_{\mathcal{L}_1},L_1)$ must be given. This theory initially accounts for some of the initial commitments, that is for some $(\Gamma,\alpha)\in C_1^+$ with $\Gamma\subseteq L_1$, $\alpha\in L_1$: $\Gamma\vdash_{\mathcal{L}_1}\alpha$ and/or for some $(\Gamma,\alpha)\in C_1^-$ with $\Gamma,\alpha\subseteq L_1$:  $\Gamma\nvdash_{\mathcal{L}_1}\alpha$. 

Now the actual RE-process begins. In the first step the commitments in $C_1$ are reconsidered. There may be commitments that could be rejected and others that are not likely to be rejected. For example \textit{modus ponens} is a principle that usually is not given up lightly, while some logicians reject \textit{ex contradictione quodlibet} straightaway. One (type of) argument that is up to revision, is tested against the theory $\mathcal{L}$. If an argument $(\Gamma,\alpha)$ is accepted, but the consequence relation does not hold, that is $\Gamma\nvdash_{\mathcal{L}_1}\alpha$, the argument (type) is rejected. If $\Gamma\vdash_{\mathcal{L}_1}\alpha$ holds, the set of commitments stays unchanged i.e. $C_2:=C_1$. 

In step two, the theory is revised. Changes in the theory may be made fore several reasons. For instance a theory might be adapted to the commitments i.e. it gets changed so that it applies for some arguments it did not before. This will be done by modifying the consequence relation which in turn will require a new definition of this relation to keep the theory systematic. Or a theory could be changed to become more systematic. Some slight changes in the consequence relation might be condoned to achieve a more elegant or simple definition e.g. by ruling out exceptions or special cases. But a change in theory could also be more fundamental. For example, a new logical symbol might be introduced. This would require to spell out a new translation scheme and a re-formalisation of the commitments according to the new scheme. 

Step three again consists in the reconsideration of the commitments. After this, we go back to the theory and make changes if necessary and so on. The process ends if our theory accounts for our commitments, formally speaking, if for every $(\Gamma,\alpha)\in C_i^+$ $\Gamma\vdash_{\mathcal{L}_i}\alpha$ holds, and for every $(\Gamma,\alpha)\in C_i^-$ $\Gamma\vdash_{\mathcal{L}_i}\alpha$ does not hold.

\section{Problems and consequences}
After working out my account of how RE can be applied in the justification of a logical theory, I will in this last part turn to the discussion of problems concerning the application of RE-theory in the philosophy of logic. I shall also compare my account to other accounts of RE and explain why I think these accounts fail.

\subsection{Shapiro's critique}
A famous critique of the idea of applying the theory of RE to logic was brought forward by \citeA{shapiro}. I will discuss Shapiro's concerns in this section and try to respond to the problems he raised.

Shapiro claims that an RE-theorist\footnote{He often refers to `Quineans', meaning someone who claims that logical principles can be revised. He therefore takes the idea of justifying derivative logical theories in an RE-process to be a special case of `Quineanism'.} faces a dilemma when formulating an account of RE. At each stage, the theory has to be tested against the commitments (or the other way round). So, for example, it must be tested whether or not the theory `\textit{coheres with}' \cite[p.~346]{shapiro} the commitments. However, concepts like `coherence' are logical terms. So the RE-theorist either (i) presupposes at least some logical principles that cannot be revised, or (ii) applies the logical principles articulated in the logical theory developed at the current RE-stage. But, as Shapiro argues, (ii) would make it trivial to attain RE: if the logical theory that tells us, whether or not a logical theory accounts for an argument or not, were up to revision, a sentence like `The theory $T$ is not in accord with the invalidity of the Argument $A$' \cite[p.~346]{shapiro} could simply be rejected, because it is itself a sentence of the logical theory that is up to revision. An RE could therefore `very easily' be established by simply rejecting all sentences of this form. \cite[p~346-367]{shapiro}.

In later works on RE and logic, both `horns of the dilemma' have been accepted. Resnik, who was addressed directly by Shapiro's critique, opted for (ii) and accepted the triviality of attaining RE \cite[p.~193]{resnik2}. According to Resnik, whether or not a logical theory that was developed by revising an old one, is accepted, is then more of a sociological question \cite[p.~190]{resnik2}.

In contrast, \citeA{peregrinsvoboda} accept the existence of a background logic an thereby go for option (i). They think of the RE-process as a method to flesh out and systematise a `proto-logic' \cite[p.~97]{peregrinsvoboda} that already is contained in everyday reasoning. \cite[p.~95-105]{peregrinsvoboda}

%If we go for option (i), 
We might then be compelled to accept, that the room of rational possible positions in the philosophy of logic is quite limited. For example, if we have to presuppose a well defined notion of coherence, as Shappiro suggests, we would most likely have to accept the LNC, for consistency is normally taken to be a necessary condition for coherence. Paraconsistent logical theories might then be ruled out \textit{a priori} as a rational logical theory. But why should we believe that there are such a priori principles? 

This is, of course, a question that gives raise to an extensive debate which cannot be discussed here. However, I think that the assumption that our language contains a substantial proto-logic is dubious. For example, in questions concerning the LNC our language is quite ambiguous. This is a fact regularly stressed by dialetheists; our natural language obviously contains a truth predicate and nearly everyone accepts that there is a prima facie plausibility to the argument that establishes the contradictory nature of the liar sentence. However, at least in some sense of the word `truth', it seems unlikely that a sentence is true and false at the same time. Whether or not the LNC holds, is a \textit{decision that has to be made} while developing a logical theory. But if it is not clear whether the LNC holds or not, we cannot decide whether coherence entails consistency. We could then try to give an account of coherence that does not entail consistency. But discarding the LNC would certainly force us to discard the disjunctive syllogism (for if $p\vee q$ is true and $p$ is false, it does not follow that $p$ is not true -- it can be true \textit{and} false -- and therefore it does not follow that $q$ must be true for the disjunction to be true) and the principle of explosion. This in turn would give rise to a notion of deductive inference that differs significantly from the one we would obtain by holding to the LNC in the first place. Furthermore, giving up the disjunctive syllogism arguably is not backed by whatever proto-logic one may find in everyday language. 

So if there is a proto-logic in everyday language it must be very weak. One may, for example, argue, that every speaker of a natural language would accept \textit{modus ponens} and that arguments relying on this rule are valid a priori. This may well be true (even though dialetheists will probably have to give a more differentiated account of this principle, as mentioned in section \ref{sec:logic} on page \pageref{sec:logic}). However, \textit{modus ponens} is certainly not enough to define such `thick' logical concepts like coherence. I therefore doubt that there are any extensive logical principles in our language that would suffice to give a substantial background logic in an RE-process.

That there are no strong a priori concepts in everyday language is due to the absence of clear logical \textit{principles} or logical \textit{laws} in our language. Nevertheless, there are certain logical notions that must in some sense be prior to an RE-process. These are the notions discussed in the previous sections, like `argument', `consequence relation' and `logical theory'. These notions define the process of developing a logical theory itself and must be fixed before the process. Otherwise, it would not even be clear what the agent carrying out the RE-process is doing. But the terms as they are understood here are very general and do not imply any logical laws as Shapiro seems to demand them to be a background theory. They can therefore be accepted by standard-logic theorists, dialetheists, intuitsionists and others.

If we take these considerations seriously, we are forced to embrace option (ii). But can we then avoid such a radical `anything-goes' approach as Resnik suggested? I think we can. Resnik does not seem to take into account that there are certain standards of rationality in the philosophical debate on different logical theories. When he writes that we may revise logic 
`[w]henever [we] feel that it is a reasonable way to respond to felt tensions in our conceptual scheme and think [our] proposal has a reasonable chance of being considered' \cite[p.~190]{resnik2}, he seems to imply that there are no intersubjective standards for the development of a logical theory at all. However, this does not seem to be the case. Even if we do not believe that there are unchangeable objective standards, it must be admitted that there currently are \textit{de facto} standards. I gave my account of what I think are four fundamental principles of a reasonable RE-process for a logical theory in section \ref{sec:spelling_out}. According to these principles, not just any process that resembles an RE-process is a rational process. Meeting certain common principles of rationality is what makes it possible to spell out an arguably rational process to justify a certain logical theory. As argued above, spelling out such a process for a logical theory is a way of justifying this theory. 

Let us return to the original concerns, brought forward by Shapiro. First, notice that his objection does not apply to the account of RE given here. The core of his critique was that the finding that the theory does not accord with the invalidity of a certain argument could simply be rejected. However, this is not possible in the framework developed here. As argued above, an argument is either accepted or not accepted but not both, and the consequence relation between a set of premises and a conclusion either holds or does not hold, but, again, not both. It is not a question of logic whether the theory accounts for an argument or not. Facts like this are simply not up to revision. But then, Shapiro's objection seems to reduce to the concern that in an RE-process everything could be done to reach an RE. However, as we have seen, this is not the case: a rational RE-process is guided by principles of rationality. 

So, what can be done in an RE-process is restricted by a background theory of what a logical consequence relation is, and how it behaves. Furthermore, there are principles of rationality that guide the RE-process. For this reason, the theory of RE I sketched here, may be called a `\textit{wide}' RE. However, this background theory \textit{guides} the process and does not change. In fact, it is important to my account of RE that the background theory does not change during the process for it has to be shared among the participants of the philosophical debate on different logical theories. An RE-process is carried out to show that a logical theory can be the outcome of a development process that is rational according to the shared background theory of rationality and is, therefore, itself a rational theory.

To conclude this final section, I will point out another difference between my theory of RE and theories like the one brought forward by \citeauthor{resnik0} or \citeauthor{peregrinsvoboda}. They have, as I will argue, a problematic understanding of the purpose of an RE-process.

\subsection{RE is not a research method}
A common conception of RE is that it can be used to \textit{develop} a logical theory. This seems to be what Resnik has in mind when he writes:

\begin{quote}
    The method logicians use when constructing systems for codifying correct reasoning or notions of logical necessity and possibility is the method of wide reflective equilibrium [...]. \cite[p.~159]{resnik0}
\end{quote}

The same idea was formulated by \citeA{peregrinsvoboda}
\begin{quote}
    We can, to some extent, also see reflective equilibrium as a methodological principle that may (or should) guide a certain theory-forming process. \cite[p.~95]{peregrinsvoboda}
\end{quote}

I do not agree with this conception of RE for two reasons. The first is, as I already argued, that it does not matter how a certain logical theory was developed to answer the question whether it can be shown to be rational. The question that has to be asked is, \textit{why does a person believe} accepts this theory. Carrying out (or at least sketching) an RE-process is a way of systematically lay out the reasons and considerations that led to a certain theory, along with the trade-offs that have to be made in this turn and, again, the reasons why they were made. This process does not have to have anything in common with the process that led to the development of the theory.

Second, the framework of an RE-process I gave here is quite complex, especially the idea to keep track of all the commitments that are relevant to the process. This is not a problem if the rationality of a certain logical theory is at question. Most likely the focus will, in such a case, lay only on some aspects of the theory and the commitments will be quite overseeable. Furthermore, we already know where the process should lead us, and which commitments will be relevant to consider. But when a theory is developed, the number of commitments is likely to be unmanageable. Furthermore, new commitments may appear in the process, for example because of new technical insights. It is unlikely that a human being can effectively develop a logical theory from scratch by carrying out an RE-process in the way I described it in section \ref{sec:spelling_out}. To argue that RE is a method to develop a logical theory requires a very broad and unspecific understanding of RE. Such an understanding of RE may lead `some philosophers [to point] out [that] the ‘reflective equilibrium strategy’ is not far removed from the ordinary scientific routine' \cite{peregrinsvoboda}. I do not find such a conception of RE to be convincing. Obviously, there are a lot of logical theories that were developed in an `ordinary scientific routine' and the question remains whether or not these are \textit{rational} logical theories. Not just any process of trying to make a theory account for the commitments by working back and forth between the theory and the commitments does lead to rational beliefs. As I argued, in an RE-process that is carried out to show that a logical theory is rational, certain rules and principles of rationality have to be considered and central terms have to be fixed.

\section{Conclusion}
The three questions I posed in the introduction can now be answered. I argued that a logical theory must be understood as a normative theory, concerned with our reasoning. It includes an exact definition of a logical consequence relation and the GGTP gives us the abstract form of such a definition. This is a very general account of logic that can be accepted by adherents of such different philosophical positions as dialetheism, intuitionism and traditional standard logic. This answers the first question -- the question of what a general and, in some way neutral, definition of logic could look like.

The second question concerned the problem that most theories of rationality presuppose logical notions. I argued that, when it comes to logical theories, my account of RE is a more promising account of rationality than foundationalist or coherentist accounts. Foundationalism must assume that there is only one rational logical theory and can therefore not account for moderate pluralism concerning logical consequence. Coherentism, on the other hand, seems to allow for many irrational theories. I suggested that one way to show that a logical theory is a rational theory, is to show that the theory can be the result of a rational development process, and this can be done by carrying out an RE-process. By doing this, the decisions and motives that lead to a certain logical theory can be made transparent.

The third question asked for concepts that must be shared among the participants of the debate on different logical theories. I argued that there have to be some shared logical terms and that we have to consider some intersubjectively accepted principles of rationality when we carry out an RE-process. Shared terms are crucial to have a fruitful debate. To discuss logical theories, there has to be agreement on what a logical theory is. As mentioned above, I specified what I think a logical theory is, and I did so in a way that can be accepted by all the opponents in the debate. 

The principles of rationality guide the RE-process and motivate the steps that take us from an RE-stage to the next one. If a process that follows these principles results in a certain logical theory, this theory is justified. This does not mean that we need to know how a logical theory was developed, to decide whether it is a rational theory or not. It suffices to show that the theory is the outcome of a process involving rational considerations and decisions in the form of an RE-process.

This account of RE differs from other accounts that take RE to be a method to develop a logical theory. However, it allows us to explain why not every logical theory can be the outcome of an RE-process: every step has to be in accordance with the principles of rationality. These principles do not strictly determine the outcome of the process and this explains why there are different rational logical theories.

\newpage
\bibliographystyle{apacite}
\bibliography{references}

\end{document}



\begin{figure}[h!]
    \centering
    \begin{tikzpicture}
        \draw (0*2.5+2,1) -- (0*2.5+2,0) -- (0*2.5,0) -- (0*2.5,1) -- cycle;
        \draw (0*2.5+1,.1) -- (0*2.5+1,.9) --cycle;
        \draw [->,thick] (0*2.5+2,.5) -- (0*2.5+2.5,.5); 
        \node at (0*2.5+.5,.5) {$\mathcal{C}_1$};
        \node at (0*2.5+1.5,.5) {$\mathcal{L}_1$};
        \draw [->,thick] (-1*2.5+2,.5) -- (-1*2.5+2.5,.5);
        
        \draw (1*2.5+2,1) -- (1*2.5+2,0) -- (1*2.5,0) -- (1*2.5,1) -- cycle;
        \draw (1*2.5+1,.1) -- (1*2.5+1,.9) --cycle;
        \draw [->,thick] (1*2.5+2,.5) -- (1*2.5+2.5,.5); 
        \node at (1*2.5+.5,.5) {$\mathcal{C}_2$};
        \node at (1*2.5+1.5,.5) {$\mathcal{L}_1$};
        
        \node [font=\large] at (2.9+2.5,.25) {...};
        \draw [->,thick] (2.5*2.5-.5,.5) -- (2.5*2.5,.5);
        
        \draw (2.5*2.5+2,1) -- (2.5*2.5+2,0) -- (2.5*2.5,0) -- (2.5*2.5,1) -- cycle;
        \draw (2.5*2.5+1,.1) -- (2.5*2.5+1,.9) --cycle;
        \draw [->,thick] (2.5*2.5+2,.5) -- (2.5*2.5+2.5,.5); 
        \node at (2.5*2.5+.5,.5) {$\mathcal{C}_i$};
        \node at (2.5*2.5+1.5,.5) {$\mathcal{L}_i$};
        
        \draw (3.5*2.5+2,1) -- (3.5*2.5+2,0) -- (3.5*2.5,0) -- (3.5*2.5,1) -- cycle;
        \draw (3.5*2.5+1,.1) -- (3.5*2.5+1,.9) --cycle;
        \draw [->,thick] (3.5*2.5+2,.5) -- (3.5*2.5+2.5,.5); 
        \node at (3.5*2.5+.5,.5) {$\mathcal{C}_i$};
        \node at (3.5*2.5+1.5,.5) {$\mathcal{L}_{i+1}$};
        
        \draw [->,thick,dashed] (0*2.5+2.25,4.5) -- (0*2.5+2.25,3.5);
        \node at (0*2.5+2.25,4.7) {proven by using inference rule $R$};
        \draw [->,thick,dashed] (0*2.5+2.25,3) -- (0*2.5+2.25,2);
        \node at (0*2.5+2.25,1.7) {\textit{accept} $(\Gamma,\alpha)$};
        \draw [->,thick,dashed] (0*2.5+2.25,1.5) -- (0*2.5+2.25,.5);
        \node at (0*2.5+2.25,3.2) {$\Gamma\vdash_{\mathcal{L}_1}\alpha$};
        
        \draw [->,thick,dashed] (2.5*2.5+2.25,1.5) -- (2.5*2.5+2.25,.5);
        \node at (2.5*2.5+2.25,1.7) {\textit{change} to $\vdash_{\mathcal{L}_{i+1}}$ with $\Gamma'\nvdash_{\mathcal{L}_{i+1}}\alpha'$ by rejecting $R$};
        \draw [->,thick,dashed] (2.5*2.5+2.25,3) -- (2.5*2.5+2.25,2);
        \node at (2.5*2.5+2.25,3.2) {$\Gamma'\vdash_{\mathcal{L}_i}\alpha'$ but $(\Gamma',\alpha')\in\mathcal{C}_i^-$};
        \draw [->,thick,dashed] (2.5*2.5+2.25,4.5) -- (2.5*2.5+2.25,3.5);
        \node at (2.5*2.5+2.25,4.7) {proven by using inference rule $R$};
        
        \node at (5,5.5) {where $(\Gamma,\alpha)\in\bigcap_{j\in\{2,...,i\}}\mathcal{C}_j^+$};
\end{tikzpicture}
    \caption{Problematic RE-process}
    \label{fig:problem1}
\end{figure}
\section*{Dispo}
The goal of my essay is to develop an account of reflective equilibrium that can be applied in the discussion about derivative logics. To do this, several questions must be answered
\begin{enumerate}
    \item What is this thing that we call logic which is developed or legitimised in an RE process?
    \item How can the challenge brought up by \citeA{shapiro} be answered e.g. how can it not be trivial to reach RE in an RE-process that is not bonded by a fixed logical theory?\begin{enumerate}
        \item What exactly does it mean that the logic developed at an RE-stage is used to determine whether this stage is an RE?
    \end{enumerate}
    \item What is a reflective equilibrium and how can it be specified without presuming too many logical notions and concepts (and thereby exclude interesting logical theories from being examined by the RE-approach).
    \item How must an RE-process be understood and how should it be carried out? What are the rules governing such a process if these rules cannot be `logical rules' in the usual sense?
    \item How and how far can an RE be normative?
    \item What to do in special cases? For example, what is the right thing to do if, in face of a very plausible commitment, in an RE-stage the rejection of an inference rule seems the best thing to do, where this very inference rule is the reason the the commitment was accepted in the first place?
    \item Are there more special cases like the one before which must be treated sperately?
    \item How can the developed account of RE be applied to dialetheist logic?
    \item What is part of the RE-process? Is it only the formal system or are the formal and the philosophical aspects closely intertwined?
    \item How can there be reasonable dispute? In other words, how can there be different views of logical consequence that can be hold ratinoally?
\end{enumerate}

There are some `leading ideas' in this discussion:
\begin{enumerate}
    \item The view of logic being taken here is a pluralist one. This might be seen as a pragmatic choice, I will not argue for logical pluralism.
    \item The interpretation of RE will mainly be concerned with the origins of the idea in \citeA{Goodman}. I will try to stay close to his description of the process.
    \item Arguing is \textit{doing something}. This explains the normative character of logic: it tells us what we should or should not do by classifying arguments. Valid arguments are arguments we should accept and invalid arguments are arguments we should not accept.
    \item Here are some key point of this essay \begin{enumerate}
        \item Reason alone does not determine which logical theory is the best. Two rational persons might disagree on the notion of logical consequence. However, principles of rationality narrow the scope of which logical theory can count as rational theories.
        \item A theory developed in a process in which every step was rational is a rational theory.
    \end{enumerate}
\end{enumerate}
The structure of the essay will be as follows:

To start off, the general idea of reflective equilibrium will be motivated and put in to a historical context. This will include some general remarks about the problems of derivative logics and a brief discussion of the relevant passages in \citeA{Goodman}. Furthermore some central terms such as `RE-stage' and `RE-process' will be introduced. (2-3 pages)

In the second part I will revisit the problem of derivative logics and address question 1. (above). It will be argued that the question of what `a logic $T$' or `a logical theory $T$' is, can be reduced to the question, how the entailment relation of $T$ is used. This finishes the preliminary work. (1.5-2 pages)

In the third part, the debate about the application of RE in philosophy of logic will be discussed. This will include a discussion of Texts like \citeA{resnik1} and \citeA{resnik2}, \citeA{shapiro} and \citeA{BrunLogic}. I will follow Brun in defending reflective equilibrium as an adequate method in the undertaking of justifying a logical theory. In addition, I will argue against \citeA{resnik2} and \citeA{shapiro} (question 2.) by criticising it as a much too weak account of RE. There will also be a brief discussion of \citeA{peregrinsvoboda} and their view of RE as a purely descriptive theory which will be the starting point of my positive work (questions 3. and 4.). (4-5 pages)

In the fourth part I will be concerned with the question of how RE can be normative. I will argue that RE can play a crucial role at explaining a certain view in logic. Further I will argue that a logical theory, explained in a sufficiently regularised RE-process, should be seen as a reasonable position to people who accept the rules governing the RE process. The upshot will be that a logical theory is reasonable if it is the result of an RE-process (ending up in an RE) which was governed by rules considered to be reasonable (question 5.). (2-3 pages)

In the fifth part I will spell out some technicalities. An abstract `formalisation' of an RE-process will be developed. This is done to sum up the preceding considerations and to have a precise framework to address further problems. (2-3 pages)

In the sixth part I will turn to the question of `special cases'. I do not know exactly how to deal with the problem mentioned in question 6 and how to answer question 7 but these will be the central topics of this section. (2-4 pages)

In the seventh and last part of of this essay will be a (pseudo-)case study of dialetheist logic from an RE-point of view. (2-4 pages)

\section*{A problem}
Consider a process like the one shown in Figure \ref{fig:problem1}. The Argument $(\Gamma,\alpha)$ is accepted on the basis of an inference rule that is rejected in a later stage of the process (by changing the theory). Nevertheless the argument is held to be valid the whole time. It seems like there is `no reason' anymore to accept the validity of $(\Gamma,\alpha)$. So is is likely that the validty of $(\Gamma,\alpha)$ should be rejected. But how can we be sure that the acceptance of $(\Gamma,\alpha)$ did not impact other decisions made in the stages before $(\mathcal{C}_i,\mathcal{L}_i)$? Actually it is likely that the acceptance of $(\Gamma,\alpha)$ led to the theory $\mathcal{L}_i$ for the next theory in the RE-process is meant to be a systematisation of the commitments at the current stage. So the theory that was revised by rejecting the inference rule $R$ might be a direct result of accepting $(\Gamma,\alpha)$ which itself was motivated by inference rule $R$. Considering this we are obviously not out of the woods by simply rejecting the validity of $(\Gamma,\alpha)$ in a next step. What to do then?

There are \textit{prima facie} two options: we could go back to the step where we accepted $(\Gamma,\alpha)$, reject the validity of the argument and continue with the process. Notice that it still could be accepted at a later stage in the process maybe it will not lead to the paradoxical situation anymore then.

